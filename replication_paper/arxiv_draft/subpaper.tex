\documentclass{article}

\usepackage{arxiv}

\usepackage[utf8]{inputenc} % allow utf-8 input
\usepackage[T1]{fontenc}    % use 8-bit T1 fonts
\usepackage{lmodern}        % https://github.com/rstudio/rticles/issues/343
\usepackage{hyperref}       % hyperlinks
\usepackage{url}            % simple URL typesetting
\usepackage{booktabs}       % professional-quality tables
\usepackage{amsfonts}       % blackboard math symbols
\usepackage{nicefrac}       % compact symbols for 1/2, etc.
\usepackage{microtype}      % microtypography
\usepackage{graphicx}

\title{Introducing ICBe: Very High Recall and Precision Event Extraction
from Narratives about International Crises}

\author{
    Rex W. Douglass
   \\
    University of California, San Diego \\
   \\
  \texttt{} \\
   \And
    Thomas Leo Scherer
   \\
    University of California, San Diego \\
   \\
  \texttt{} \\
   \And
    J. Andrés Gannon
   \\
    Vanderbilt University \\
   \\
  \texttt{} \\
   \And
    Erik Gartzke
   \\
    University of California, San Diego \\
   \\
  \texttt{} \\
   \And
    Jon Lindsay
   \\
    Georgia Institute of Technology \\
   \\
  \texttt{} \\
   \And
    Shannon Carcelli
   \\
    University of Maryland \\
   \\
  \texttt{} \\
   \And
    Jonathan Wilkenfeld
   \\
    University of Maryland \\
   \\
  \texttt{} \\
   \And
    David M. Quinn
   \\
    University of Maryland \\
   \\
  \texttt{} \\
   \And
    Catherine Aiken
   \\
    Georgetown University \\
   \\
  \texttt{} \\
   \And
    Jose Miguel Cabezas Navarro
   \\
    Universidad Mayor \\
   \\
  \texttt{} \\
   \And
    Neil Lund
   \\
    University of Maryland \\
   \\
  \texttt{} \\
   \And
    Egle Murauskaite
   \\
    University of Maryland \\
   \\
  \texttt{} \\
   \And
    Diana Partridge
   \\
    University of Maryland \\
   \\
  \texttt{} \\
  }


% tightlist command for lists without linebreak
\providecommand{\tightlist}{%
  \setlength{\itemsep}{0pt}\setlength{\parskip}{0pt}}


% Pandoc citation processing
\newlength{\cslhangindent}
\setlength{\cslhangindent}{1.5em}
\newlength{\csllabelwidth}
\setlength{\csllabelwidth}{3em}
\newlength{\cslentryspacingunit} % times entry-spacing
\setlength{\cslentryspacingunit}{\parskip}
% for Pandoc 2.8 to 2.10.1
\newenvironment{cslreferences}%
  {}%
  {\par}
% For Pandoc 2.11+
\newenvironment{CSLReferences}[2] % #1 hanging-ident, #2 entry spacing
 {% don't indent paragraphs
  \setlength{\parindent}{0pt}
  % turn on hanging indent if param 1 is 1
  \ifodd #1
  \let\oldpar\par
  \def\par{\hangindent=\cslhangindent\oldpar}
  \fi
  % set entry spacing
  \setlength{\parskip}{#2\cslentryspacingunit}
 }%
 {}
\usepackage{calc}
\newcommand{\CSLBlock}[1]{#1\hfill\break}
\newcommand{\CSLLeftMargin}[1]{\parbox[t]{\csllabelwidth}{#1}}
\newcommand{\CSLRightInline}[1]{\parbox[t]{\linewidth - \csllabelwidth}{#1}\break}
\newcommand{\CSLIndent}[1]{\hspace{\cslhangindent}#1}

\usepackage[utf8]{inputenc}
\usepackage{pifont}
\usepackage{newunicodechar}
\newunicodechar{✓}{\ding{51}}
\newunicodechar{✗}{\ding{55}}
\usepackage{array}
\usepackage{ctable}
\usepackage{booktabs}
\usepackage{longtable}
\usepackage{array}
\usepackage{multirow}
\usepackage{wrapfig}
\usepackage{colortbl}
\usepackage{pdflscape}
\usepackage{tabu}
\usepackage{threeparttable}
\usepackage{threeparttablex}
\usepackage[normalem]{ulem}
\usepackage{makecell}
\usepackage{titlesec}
\usepackage[parfill]{parskip}
\usepackage{makecell}
\usepackage{graphicx}
\usepackage{setspace}
\usepackage{cellspace}
\setlength\cellspacetoplimit{0.8ex}
\renewcommand{\arraystretch}{0.8}
\AtBeginEnvironment{lltable}{\singlespacing}
\usepackage{subcaption}
\usepackage{atbegshi}% http://ctan.org/pkg/atbegshi
\usepackage{float}
\usepackage[algo2e]{algorithm2e}
\begin{document}
\maketitle


\begin{abstract}
How do international crises unfold? We conceptualize of international
relations as a strategic chess game between adversaries and develop a
systematic way to measure pieces, moves, and gambits accurately and
consistently over a hundred years of history. We introduce a new
ontology and dataset of international events called ICBe based on a very
high-quality corpus of narratives from the International Crisis Behavior
(ICB) Project. We demonstrate that ICBe has higher coverage, recall, and
precision than existing state of the art datasets and conduct two
detailed case studies of the Cuban Missile Crisis (1962) and
Crimea-Donbas Crisis (2014). We further introduce two new event
visualizations (event icongraphy and crisis maps), an automated
benchmark for measuring event recall using natural language processing
(sythnetic narratives), and an ontology reconstruction task for
objectively measuring event precision. We make the data, online
appendix, replication material, and visualizations of every historical
episode available at a companion website www.crisisevents.org and the
github repository.
\end{abstract}

\keywords{
    Diplomacy
   \and
    War
   \and
    Crises
   \and
    International Affairs
   \and
    Computational Social Science
  }

\twocolumn

If we could record every important interaction between countries in all
of diplomacy, military conflict, and international political economy,
how much unique information would this chronicle amount to, and how
surprised would we be to see something new? In other words, what is the
entropy of international relations? This record could in principle be
unbounded, but the central conceit of social science is that there are
structural regularities that limit what actors can do, their best
options, and even which actors are likely to survive (1, 2). If so, then
these events can be systematically measured, and accordingly, massive
effort is expended in social science attempting to record these
regularities.\footnote{See work on crises (3, 4), militarized disputes
  (5--7), wars (8, 9), organized violence (10, 11), political violence
  (12), sanctions (13), trade (14), and international agreements
  (15--17), dispute resolution (17, 18), and diplomacy (19, 20).} Thanks
to improvements in natural language processing, more open-ended efforts
have begun to capture entire unstructured streams of events from
international news reports.\footnote{See (21); (22); (23); (24); (25);
  (26). On event-extraction from images and social-media see (27) and
  (28).} How close these efforts are to accurately measuring all or even
most of what is essential in international relations is an open
empirical question, one for which we provide new evidence here.

Our contribution is a high coverage ontology and event dataset for key
historical episodes in 20th and 21st-century international relations. We
develop a large, flexible ontology of international events with the help
of both human coders and natural language processing. We apply it
sentence-by-sentence to an unusually high-quality corpus of historical
narratives of international crises (1, 29--32). The result is a new
lower bound estimate of how much actually happens between states during
pivotal historical episodes. We then develop several methods for
objectively gauging how well these event codings reconstruct the
information contained in the original narrative. We conclude by
benchmarking our event codings against several current state-of-the-art
event data collection efforts. We find that existing systems produce
sequences of events that do not contain enough information to
reconstruct the underlying historical episode. The underlying
fine-grained variation in international affairs is unrecognizable
through the lens of current quantification efforts.

This is a measurement paper that makes the following argument --- there
is a real-world unobserved latent concept known as international
relations, we propose a method for systematically measuring it, we
successfully apply this method producing a new large scale set of
measurements, those measurements exhibit several desirable kinds of
internal and external validity, and those measurements out-perform other
existing approaches. The article organizes that argument into eight
sections: task definition; corpus; priors/existing state of the art;
ICBe coding process; internal consistency; case study selection; recall;
and precision. A final section concludes.

\hypertarget{task-definition}{%
\section*{Task Definition}\label{task-definition}}
\addcontentsline{toc}{section}{Task Definition}

We consider the measurement task of abstracting discrete events about a
historical episode in international relations. The easiest way to convey
the task is with an example. Figure 1 shows a narrative account of the
Cuban Missile Crisis (1962) alongside a mapping from each natural
language sentence to discrete machine readable abstractive events.
Formally, a historical episode, \(H\), is demarcated by a period of time
\([T_{start}, T_{end}] \in T\), a set of Players \(p \in P\), and a set
of behaviors they undertook during that time \(b \in B\). International
Relations, \(IR\), is the system of regularities that govern the
strategic interactions that world actors make during a historical
episode, given their available options, preferences, beliefs, and
expectations of choices made by others. We observe neither \(H\) nor
\(IR\) directly. Rather the Historical Record, \(HR\), produces
documents \(d \in D\) containing some relevant and true (as well as
irrelevant and untrue) information about behaviors that were undertaken
recorded in the form of unstructured natural language text. The task is
to combine informative priors about \(IR\) with an unstructured corpus
\(D\) to produce a series of structured discrete events, \(e \in E\),
that have high coverage, precision, and recall over what actually took
place in history, \(H\).

\hypertarget{format}{%
\section*{Conclusion}\label{format}}
\addcontentsline{toc}{section}{Conclusion}

We investigated event abstraction from narratives describing key
historical episodes in international relations. We synthesized a prior
belief about the latent unobserved phenomena that drive these events in
international relations and proposed a mapping to observable concepts
that enter into the observed historical record. We designed an ontology
with high coverage over those concepts and developed a training
procedure and technical stack for human coding of historical texts.
Multiple validity checks find the resulting codings have high internal
validity (e.g.~intercoder agreement) and external validity
(i.e.~matching source material in both micro-details at the sentence
level and macro-details spanning full historical episodes). Further,
these codings perform much better in terms of recall, precision,
coverage, and overall coherence in capturing these historical episodes
than existing event systems used in international relations.

We release several open-source products along with supporting code and
documentation to further advance the study of IR, event extraction, and
natural language processing. The first is the International Crisis
Behavior Events (ICBe) dataset, an event-level aggregation of what took
place during the crises identified by the ICB project. These data are
appropriate for statistical analysis of hard questions about the
sequencing of events (e.g.~escalation and de-escalation of conflicts).
Second, we provide a coder-level disaggregation with multiple codings of
each sentence by experts and undergrads that allows for the introduction
of uncertainty and human interpretation of events. Further, we release a
direct mapping from the codings to the source text at the sentence level
as a new resource for natural language processing. Finally, we provide a
companion website that incorporates detailed visualizations of all of
the data introduced here (www.crisisevents.org).

\hypertarget{format}{%
\section*{Acknowledgements}\label{format}}
\addcontentsline{toc}{section}{Acknowledgements}

We thank the ICB Project and its directors and contributors for their
foundational work and their help with this effort. We make special
acknowledgment of Michael Brecher for helping found the ICB project in
1975, creating a resource that continues to spark new insights to this
day. We thank the many undergraduate coders for their patience and
dedication. Thanks to the Center for Peace and Security Studies and its
membership for comments. Special thanks to Rebecca Cordell, Philip
Schrodt, Zachary Steinert-Threlkeld, and Zhanna Terechshenko for
generous feedback. Thank you to the cPASS research assistants that
contributed to this project: Helen Chung, Daman Heer, Syeda ShahBano
Ijaz, Anthony Limon, Erin Ling, Ari Michelson, Prithviraj Pahwa, Gianna
Pedro, Tobias Stodiek, Yiyi `Effie' Sun, Erin Werner, Lisa Yen, and
Ruixuan Zhang. This project was supported by a grant from the Office of
Naval Research {[}N00014-19-1-2491{]} and benefited from the Charles
Koch Foundation's support for the Center for Peace and Security Studies.

\hypertarget{format}{%
\section*{Author Contributions}\label{format}}
\addcontentsline{toc}{section}{Author Contributions}

Conceptualization: R.W.D., E.G., J.L.; Methodology: R.W.D., T.L.S.;
Software: R.W.D.; Validation: R.W.D., T.L.S.; Formal Analysis: R.W.D.,
T.L.S.; Investigation: S.C., R.W.D., J.A.G., C.K., N.L., E.M., J.M.C.N.,
D.P., D.Q., J.W.; Data Curation: R.W.D., D.Q., T.L.S., J.W.; Writing -
Original Draft: R.W.D., T.L.S.; Writing - Review \& Editing: R.W.D.,
J.A.G., E.G., T.L.S.; Visualization: R.W.D., T.L.S.; Supervision: E.G.;
Project Administration: S.C., R.W.D., J.A.G., D.Q., T.L.S., J.W.;
Funding Acquisition: E.G., J.L.

\hypertarget{format}{%
\section*{Bibliography}\label{format}}
\addcontentsline{toc}{section}{Bibliography}

\hypertarget{refs}{}
\begin{CSLReferences}{0}{0}
\leavevmode\vadjust pre{\hypertarget{ref-brecherInternationalStudiesTwentieth1999}{}}%
\CSLLeftMargin{1. }%
\CSLRightInline{Brecher M (1999) International studies in the twentieth
century and beyond: {Flawed} dichotomies, synthesis, cumulation: {ISA}
presidential address. \emph{International Studies Quarterly}
43(2):213--264.}

\leavevmode\vadjust pre{\hypertarget{ref-reiterShouldWeLeave2015}{}}%
\CSLLeftMargin{2. }%
\CSLRightInline{Reiter D (2015)
\href{https://doi.org/10.1146/annurev-polisci-053013-041156}{Should {We
Leave Behind} the {Subfield} of {International Relations}?} \emph{Annu
Rev Polit Sci} 18(1):481--499.}

\leavevmode\vadjust pre{\hypertarget{ref-brecherCrisesWorldPolitics1982}{}}%
\CSLLeftMargin{3. }%
\CSLRightInline{Brecher M, Wilkenfeld J (1982)
\href{https://doi.org/10.2307/2010324}{Crises in {World Politics}}.
\emph{World Politics} 34(3):380--417.}

\leavevmode\vadjust pre{\hypertarget{ref-beardsleyInternationalCrisisBehavior2020}{}}%
\CSLLeftMargin{4. }%
\CSLRightInline{Beardsley K, James P, Wilkenfeld J, Brecher M (2020) The
{International Crisis Behavior Project}.
doi:\href{https://doi.org/10.1093/acrefore/9780190228637.013.1638}{10.1093/acrefore/9780190228637.013.1638}.}

\leavevmode\vadjust pre{\hypertarget{ref-palmerMID5Dataset20112021}{}}%
\CSLLeftMargin{5. }%
\CSLRightInline{Palmer G, et al. (2021)
\href{https://doi.org/10.1177/0738894221995743}{The {MID5 Dataset},
2011--2014: {Procedures}, coding rules, and description}. \emph{Conflict
Management and Peace Science}:0738894221995743.}

\leavevmode\vadjust pre{\hypertarget{ref-giblerInternationalConflicts181620102018}{}}%
\CSLLeftMargin{6. }%
\CSLRightInline{Gibler DM (2018) \emph{International {Conflicts},
1816-2010: {Militarized Interstate Dispute Narratives}} ({Rowman \&
Littlefield}) Available at:
\url{https://books.google.com?id=_4VTDwAAQBAJ}.}

\leavevmode\vadjust pre{\hypertarget{ref-maozDyadicMilitarizedInterstate2019}{}}%
\CSLLeftMargin{7. }%
\CSLRightInline{Maoz Z, Johnson PL, Kaplan J, Ogunkoya F, Shreve AP
(2019) \href{https://doi.org/10.1177/0022002718784158}{The {Dyadic
Militarized Interstate Disputes} ({MIDs}) {Dataset Version} 3.0:
{Logic}, {Characteristics}, and {Comparisons} to {Alternative
Datasets}}. \emph{Journal of Conflict Resolution} 63(3):811--835.}

\leavevmode\vadjust pre{\hypertarget{ref-sarkeesResortWar181620072010}{}}%
\CSLLeftMargin{8. }%
\CSLRightInline{Sarkees MR, Wayman F (2010) \emph{Resort to war:
1816-2007} ({CQ Press}).}

\leavevmode\vadjust pre{\hypertarget{ref-reiterRevisedLookInterstate2016}{}}%
\CSLLeftMargin{9. }%
\CSLRightInline{Reiter D, Stam AC, Horowitz MC (2016)
\href{https://doi.org/10.1177/0022002714553107}{A {Revised Look} at
{Interstate Wars}, 1816--2007}. \emph{Journal of Conflict Resolution}
60(5):956--976.}

\leavevmode\vadjust pre{\hypertarget{ref-ralphsundbergUCDPGEDCodebook2016}{}}%
\CSLLeftMargin{10. }%
\CSLRightInline{Ralph Sundberg, Mihai Croicu (2016) \emph{{UCDP GED
Codebook} version 5.0} ({Department of Peace and Conflict Research,
Uppsala University}).}

\leavevmode\vadjust pre{\hypertarget{ref-petterssonOrganizedViolence19892018}{}}%
\CSLLeftMargin{11. }%
\CSLRightInline{Pettersson T, Eck K (2018)
\href{https://doi.org/10.1177/0022343318784101}{Organized violence,
1989--2017}. \emph{Journal of Peace Research} 55(4):535--547.}

\leavevmode\vadjust pre{\hypertarget{ref-raleighIntroducingACLEDArmed2010}{}}%
\CSLLeftMargin{12. }%
\CSLRightInline{Raleigh C, Linke A, Hegre H, Karlsen J (2010)
Introducing {ACLED}: An armed conflict location and event dataset:
Special data feature. \emph{Journal of peace research} 47(5):651--660.}

\leavevmode\vadjust pre{\hypertarget{ref-felbermayrGlobalSanctionsData2020}{}}%
\CSLLeftMargin{13. }%
\CSLRightInline{Felbermayr G, Kirilakha A, Syropoulos C, Yalcin E, Yotov
YV (2020) \href{https://doi.org/10.1016/j.euroecorev.2020.103561}{The
global sanctions data base}. \emph{European Economic Review}
129:103561.}

\leavevmode\vadjust pre{\hypertarget{ref-barariDemocracyTradePolicy}{}}%
\CSLLeftMargin{14. }%
\CSLRightInline{Barari S, Kim IS Democracy and {Trade Policy} at the
{Product Level}: {Evidence} from a {New Tariff-line Dataset}. 16.}

\leavevmode\vadjust pre{\hypertarget{ref-kinneDefenseCooperationAgreement2020}{}}%
\CSLLeftMargin{15. }%
\CSLRightInline{Kinne BJ (2020)
\href{https://doi.org/10.1177/0022002719857796}{The {Defense Cooperation
Agreement Dataset} ({DCAD})}. \emph{Journal of Conflict Resolution}
64(4):729--755.}

\leavevmode\vadjust pre{\hypertarget{ref-owsiakInternationalBorderAgreements2018}{}}%
\CSLLeftMargin{16. }%
\CSLRightInline{Owsiak AP, Cuttner AK, Buck B (2018)
\href{https://doi.org/10.1177/0738894216646978}{The {International
Border Agreements Dataset}}. \emph{Conflict Management and Peace
Science} 35(5):559--576.}

\leavevmode\vadjust pre{\hypertarget{ref-vabulasCooperationAutonomyBuilding2021}{}}%
\CSLLeftMargin{17. }%
\CSLRightInline{Vabulas F, Snidal D (2021)
\href{https://doi.org/10.1177/0022343320943920}{Cooperation under
autonomy: {Building} and analyzing the {Informal Intergovernmental
Organizations} 2.0 dataset}. \emph{Journal of Peace Research}
58(4):859--869.}

\leavevmode\vadjust pre{\hypertarget{ref-frederickIssueCorrelatesWar2017}{}}%
\CSLLeftMargin{18. }%
\CSLRightInline{Frederick BA, Hensel PR, Macaulay C (2017)
\href{https://doi.org/10.1177/0022343316676311}{The {Issue Correlates}
of {War Territorial Claims Data}, 1816--20011}. \emph{Journal of Peace
Research} 54(1):99--108.}

\leavevmode\vadjust pre{\hypertarget{ref-moyerWhatAreDrivers2020}{}}%
\CSLLeftMargin{19. }%
\CSLRightInline{Moyer JD, Turner SD, Meisel CJ (2020)
\href{https://doi.org/10.1177/0022343320929740}{What are the drivers of
diplomacy? {Introducing} and testing new annual dyadic data measuring
diplomatic exchange}. \emph{Journal of Peace
Research}:0022343320929740.}

\leavevmode\vadjust pre{\hypertarget{ref-sechserMilitarizedCompellentThreats2011}{}}%
\CSLLeftMargin{20. }%
\CSLRightInline{Sechser TS (2011)
\href{https://doi.org/10.1177/0738894211413066}{Militarized {Compellent
Threats}, 1918--2001}. \emph{Conflict Management and Peace Science}
28(4):377--401.}

\leavevmode\vadjust pre{\hypertarget{ref-liComprehensiveSurveySchemabased2021}{}}%
\CSLLeftMargin{21. }%
\CSLRightInline{Li Q, et al. (2021) A {Comprehensive Survey} on
{Schema-based Event Extraction} with {Deep Learning}. Available at:
\url{http://arxiv.org/abs/2107.02126} {[}Accessed September 10,
2021{]}.}

\leavevmode\vadjust pre{\hypertarget{ref-haltermanExtractingPoliticalEvents2020}{}}%
\CSLLeftMargin{22. }%
\CSLRightInline{Halterman A (2020) Extracting {Political Events} from
{Text Using Syntax} and {Semantics}.}

\leavevmode\vadjust pre{\hypertarget{ref-brandtPhoenixRealTimeEvent2018}{}}%
\CSLLeftMargin{23. }%
\CSLRightInline{Brandt PT, DOrazio V, Holmes J, Khan L, Ng V (2018)
Phoenix {Real-Time Event Data}. Available at:
\url{http://eventdata.utdallas.edu}.}

\leavevmode\vadjust pre{\hypertarget{ref-boscheeICEWSCodedEvent2015}{}}%
\CSLLeftMargin{24. }%
\CSLRightInline{Boschee E, et al. (2015) {ICEWS} coded event data.
\emph{Harvard Dataverse} 12.}

\leavevmode\vadjust pre{\hypertarget{ref-hegreIntroducingUCDPCandidate2020}{}}%
\CSLLeftMargin{25. }%
\CSLRightInline{Hegre H, Croicu M, Eck K, Högbladh S (2020) Introducing
the {UCDP Candidate Events Dataset}. \emph{Research \& Politics}
7(3):2053168020935257.}

\leavevmode\vadjust pre{\hypertarget{ref-grantOUEventData2017}{}}%
\CSLLeftMargin{26. }%
\CSLRightInline{Grant C, Halterman A, Irvine J, Liang Y, Jabr K (2017)
{OU Event Data Project}. Available at: \url{https://osf.io/4m2u7/}
{[}Accessed September 1, 2021{]}.}

\leavevmode\vadjust pre{\hypertarget{ref-zhangCASMDeepLearningApproach2019}{}}%
\CSLLeftMargin{27. }%
\CSLRightInline{Zhang H, Pan J (2019)
\href{https://doi.org/10.1177/0081175019860244}{{CASM}: {A Deep-Learning
Approach} for {Identifying Collective Action Events} with {Text} and
{Image Data} from {Social Media}}. \emph{Sociological Methodology}
49(1):1--57.}

\leavevmode\vadjust pre{\hypertarget{ref-steinert-threlkeldFutureEventData2019}{}}%
\CSLLeftMargin{28. }%
\CSLRightInline{Steinert-Threlkeld ZC (2019)
\href{https://doi.org/10.1177/0081175019860238}{The {Future} of {Event
Data Is Images}}. \emph{Sociological Methodology} 49(1):68--75.}

\leavevmode\vadjust pre{\hypertarget{ref-brecherCrisisEscalationWar2000}{}}%
\CSLLeftMargin{29. }%
\CSLRightInline{Brecher M, James P, Wilkenfeld J (2000) Crisis
escalation to war: {Findings} from the {International Crisis Behavior
Project}. \emph{What Do We Know About War}.}

\leavevmode\vadjust pre{\hypertarget{ref-wilkenfeldInterstateCrisesViolence2000}{}}%
\CSLLeftMargin{30. }%
\CSLRightInline{Wilkenfeld J, Brecher M (2000) Interstate crises and
violence: Twentieth-century findings. \emph{Handbook of war studies
II}:282--300.}

\leavevmode\vadjust pre{\hypertarget{ref-jamesWhatWeKnow2019}{}}%
\CSLLeftMargin{31. }%
\CSLRightInline{James P (2019)
\href{https://doi.org/10.1177/0738894218793135}{What do we know about
crisis, escalation and war? {A} visual assessment of the {International
Crisis Behavior Project}}. \emph{Conflict Management and Peace Science}
36(1):3--19.}

\leavevmode\vadjust pre{\hypertarget{ref-iakhnisCrisesWorldPolitics2019}{}}%
\CSLLeftMargin{32. }%
\CSLRightInline{Iakhnis E, James P (2019)
\href{https://doi.org/10.1177/0738894219855610}{Near crises in world
politics: {A} new dataset}. \emph{Conflict Management and Peace
Science}:0738894219855610.}

\end{CSLReferences}

\bibliographystyle{unsrt}
\bibliography{paper.bib}


\end{document}
